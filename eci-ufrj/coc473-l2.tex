\documentclass{homework}
\usepackage{homework}

\usepackage[brazil]{babel}

\title{COC473 - Lista 2}
\author{Pedro Maciel Xavier}
\register{116023847}
\date{22 de setemebro de 2019}

\begin{document}
	
	\maketitle
	
	\quest%%1
	
	Aqui está o trecho do código que implementa a função para o cálculo do maior autovalor e seu respectivo autovetor través do método das Potências ("\textit{Power Method}"). As funções auxiliares se encontram no código completo, no apêndice.
	
	\begin{fortran}
	function power_method(A, n, x, l) result (ok)
		implicit none
		integer :: n
		integer :: k = 0
		
		double precision :: A(n, n)
		double precision :: x(n)
		double precision :: l, ll
		
		logical :: ok
		
	!   Inicializa vetor aleatório e define primeira componente como '1'
		x(:) = rand_vector(n)
		x(1) = 1.0D0
		
		l = 0.0D0
		
		do while (k < MAX_ITER)
			ll = l
			
			x(:) = matmul(A, x)                
			
	!	    Obtém autovalor
			l = x(1)
			
	!		Obtém autovetor                
			x(:) = x(:) / l
			
	!		Verifica se a tolerância foi atendida			
			if (DABS((l - ll) / l) < TOL) then
				ok = .TRUE.
				return
			else 
				k = k + 1
				continue
			end if
		end do
		ok = .FALSE.
		return
	end function
	\end{fortran}
	 
	\quest%%2
	
	O cálculo dos autovalores pelo método de Jacobi está implementado no código abaixo. As funções auxiliares, como a que calcula a matriz de rotação de \textit{Givens} e a que gera uma matriz identidade estão no código completo, nos apêndices.
	
	\begin{fortran}
	function Jacobi_eigen(A, n, L, X) result (ok)
		implicit none
		integer :: n, i, j, u, v
		integer :: k = 0
		
		double precision :: A(n, n), L(n, n), X(n, n), P(n, n)
		double precision :: y, z
		
		logical :: ok
		
		X(:, :) = id_matrix(n)
		L(:, :) = A(:, :)
		
		do while (k < MAX_ITER)
			z = 0.0D0
			do i = 1, n
				do j = 1, i - 1
					y = DABS(L(i, j))
					
	!	            Encontrado um novo valor absoluto máximo                        
					if (y > z) then
						u = i
						v = j
						z = y
					end if
				end do
			end do
			
			if (z >= TOL) then
				P(:, :) = given_matrix(L, n, u, v)
				L(:, :) = matmul(matmul(transpose(P), L), P)
				X(:, :) = matmul(X, P)
				k = k + 1
			else
				ok = .TRUE.
				return
			end if
		end do
		ok = .FALSE.
		return
	end function	
	\end{fortran}
	
	\quest%%3
		Seja o seguinte sistema de equações $\vec{A} \vec{x} = \vec{B}$:
	
		$$
		\vec{A} = \left[\begin{array}{@{}ccc@{}}
		 3 &  2 &  0 \\
		 2 &  3 & -1 \\
		 0 & -1 &  3
		\end{array}\right]
		~
		\vec{b} = \left[\begin{array}{@{}c@{}}
		 1 \\
		-1 \\
		 1 
		\end{array}\right]
		$$
	
		\subsubquest Para obter o polinômio característico, calculamos
			\begin{align*}
				\det (\vec{A} - \lambda \vec{I}) &= \left|\begin{array}{@{}ccc@{}}
				3 - \lambda &  2 &  0 \\
				2 & 3 - \lambda & -1 \\
				0 & -1 & 3 - \lambda
				\end{array}\right|\\
				&= (3 - \lambda) \left|\begin{array}{@{}cc@{}}
				3 - \lambda & -1 \\
				-1 & 3 - \lambda
				\end{array}\right| - 2 \left|\begin{array}{@{}cc@{}}
				 2 &  0 \\
				-1 & 3 - \lambda
				\end{array}\right|\\
				&= (3 - \lambda) [(3 - \lambda)^2 - 1] - 4 (3 - \lambda)\\
				&= (3 - \lambda) [(3 - \lambda)^2 - 5]\\
				&= (3 - \lambda) (\lambda^2 - 6\lambda + 4)\\
			\end{align*}
		Buscando as raízes $\lambda_i$ deste polinômio, podemos afirmar que $\lambda = 3$ é um dos autovalores. Usando a fórmula de Bhaskara, encontramos os demais.
			\begin{align*}
				\Delta &= (-6)^2 - 4 \cdot 1 \cdot 4 = 20\\
				\implies \lambda_i &= \frac{6 \pm \sqrt{20}}{2} = 3 \pm \sqrt{5}
			\end{align*}
		Assim, $\lambda_i \in \{3 - \sqrt{5}, 3, 3 + \sqrt{5}\}$. Os autovetores $\vec{v}_i$, por outro lado, devem satisfazer
			$$ \vec{A} \vec{v}_i = \lambda_i \vec{v}_i $$
		Logo, um autovetor $\vec{v}$ da forma $\left[x, y, z\right]^\T$ obedece
			\begin{align*}
				\left[\begin{array}{@{}ccc@{}}
				3 &  2 &  0 \\
				2 &  3 & -1 \\
				0 & -1 &  3
				\end{array}\right] \left[\begin{array}{@{}c@{}}
				x \\
				y \\
				z 
				\end{array}\right] = \left[\begin{array}{@{}c@{}}
				\lambda x \\
				\lambda y \\
				\lambda z 
				\end{array}\right]
			\end{align*}
		ou seja,
			\begin{align*}
			\begin{array}{@{}ccccccc@{}}
			3x&+&2y&~&~~&=&\lambda x\\
			2x&+&3y&-&~z&=&\lambda y\\
			~~&-&~y&+&3z&=&\lambda z
			\end{array}
			\end{align*}
		de onde tiramos que
			\begin{align*}
				y &= \frac{(\lambda - 3)}{2} x\\
				z &= 2x -(\lambda - 3) y = \frac{4 - (\lambda - 3)^2}{2} x
			\end{align*}
		e com isso dizemos que todo autovetor de $\vec{A}$ tem a forma
			\begin{align*}
				\left[\begin{array}{@{}c@{}}
				1 \\
				\frac{(\lambda - 3)}{2} \\
				\frac{4 - (\lambda - 3)^2}{2}
				\end{array}\right]
			\end{align*}
		Substituindo os autovalores na relação:
			\begin{align*}
			\vec{v} \in \left\{\left[\begin{array}{@{}c@{}}
			1 \\
			\frac{-\sqrt{5}}{2} \\
			\frac{-1}{2}
			\end{array}\right],
			\left[\begin{array}{@{}c@{}}
			1 \\
			0 \\
			2
			\end{array}\right],
			\left[\begin{array}{@{}c@{}}
			1 \\
			\frac{\sqrt{5}}{2} \\
			\frac{-1}{2}
			\end{array}\right]
			\right\}~\text{ e }~
			\lambda \in \left\{3 - \sqrt{5}, 3, 3 + \sqrt{5}\right\}
			\end{align*}
		respectivamente.
		
		
		\subsubquest Como todos os autovalores são positivos $(\lambda_i > 0)$, podemos afirmar que $\vec{A}$ é positiva definida.
		
		\subsubquest O método da potência ("\textit{Power Method}") consiste em, partindo de um vetor $\vec{x}^{(0)}$, cuja primeira componente é $1$, aplicar sucessivamente a matriz $\vec{A}$ sobre o vetor, normalizando suas demais entradas, dividindo-as pelo valor da primeira a cada iteração $k$. Seja $\vec{y}^{(k)} = \vec{A} \vec{x}^{(k)}$. Assim, podemos dizer que
			\begin{align*}
				\vec{x}^{(k)} = \frac{\vec{y}^{(k-1)}}{\vec{y}_1^{(k-1)}}
			\end{align*}
		Observando as entradas da matriz, afirmamos que
			\begin{align*}
				\vec{y}_1^{(k)} &= 3 \vec{x}_1^{(k)} + 2 \vec{x}_2^{(k)}\\
				\vec{y}_2^{(k)} &= 2 \vec{x}_1^{(k)} + 3 \vec{x}_2^{(k)} - \vec{x}_3^{(k)}\\
				\vec{y}_3^{(k)} &= - \vec{x}_2^{(k)} + 3 \vec{x}_3^{(k)}
			\end{align*}
		e, portanto
			\begin{align*}
			\vec{x}_1^{(k)} &= 1\\
			\vec{x}_2^{(k)} &= \frac{2 \vec{x}_1^{(k-1)} + 3 \vec{x}_2^{(k-1)} - \vec{x}_3^{(k-1)}}{3 \vec{x}_1^{(k-1)} + 2 \vec{x}_2^{(k-1)}}\\
			\vec{x}_3^{(k)} &= \frac{- \vec{x}_2^{(k-1)} + 3 \vec{x}_3^{(k-1)}}{3 \vec{x}_1^{(k-1)} + 2 \vec{x}_2^{(k-1)}}
			\end{align*}
		Para calcular o valor de $\vec{x}$, tomemos o limite de $\vec{x}^{(k)}$ quando $k \to \infty$, sobre cada componente
			\begin{align*}
				\vec{x}_1 &= \Lim{k \to \infty} \vec{x}_1^{(k)} = 1\\
				\vec{x}_2 &= \Lim{k \to \infty} \vec{x}_2^{(k)} = \frac{2 \Lim{k \to \infty} \vec{x}_1^{(k-1)} + 3 \Lim{k \to \infty} \vec{x}_2^{(k-1)} - \Lim{k \to \infty} \vec{x}_3^{(k-1)}}{3 \Lim{k \to \infty} \vec{x}_1^{(k-1)} + 2 \Lim{k \to \infty} \vec{x}_2^{(k-1)}}\\
				\vec{x}_3 &= \Lim{k \to \infty} \vec{x}_3^{(k)} = \frac{- \Lim{k \to \infty} \vec{x}_2^{(k-1)} + 3 \Lim{k \to \infty} \vec{x}_3^{(k-1)}}{3 \Lim{k \to \infty} \vec{x}_1^{(k-1)} + 2 \Lim{k \to \infty} \vec{x}_2^{(k-1)}}
			\end{align*}
		Como para toda sequência convergente $a_k \in \mathbb{R}$, $\Lim{k \to \infty} a_k = L \implies \Lim{k \to \infty} a_{k - 1} = L$, segue que
			\begin{align*}
				\vec{x}_1 &= 1\\
				\vec{x}_2 &= \frac{2 + 3 \vec{x}_2 - \vec{x}_3}{3 + 2 \vec{x}_2}\\
				\vec{x}_3 &= \frac{- \vec{x}_2 + 3  \vec{x}_3}{3 + 2 \vec{x}_2}
			\end{align*}
		Consequentemente,
			\begin{align*}\left\{\begin{array}{@{}ccc@{}}
			3 \vec{x}_2 + 2 \vec{x}_2^2 &=& 2 + 3 \vec{x}_2 - \vec{x}_3\\
			3 \vec{x}_3 + 2 \vec{x}_2 \vec{x}_3 &=& - \vec{x}_2 + 3 \vec{x}_3
			\end{array}\right. \implies \left\{\begin{array}{@{}ccc@{}}
			2 \vec{x}_2^2 &=& 2 - \vec{x}_3\\
			2 \vec{x}_2 \vec{x}_3 &=& - \vec{x}_2
			\end{array}\right. \implies \left\{\begin{array}{@{}ccc@{}}
			\vec{x}_2 &=& \frac{\sqrt{5}}{2}\\
			\vec{x}_3 &=& - \frac{1}{2}
			\end{array}\right.
			\end{align*}
		Portanto, o autovetor $\vec{v}_\text{max}$ associado ao autovalor de maior módulo é
		\begin{align*}
			\vec{v}_\text{max} = \left[\begin{array}{@{}c@{}}
			1 \\
			\frac{\sqrt{5}}{2} \\
			- \frac{1}{2}
			\end{array}\right]
		\end{align*}
		e, por construção, o maior autovalor $\lambda_\text{max}$ é dado por
			$$ \lambda_\text{max} = 3 \vec{x}_1 + 2 \vec{x}_2 = 3 + \sqrt{5} $$
			
		\subsubquest Segue o passo-a-passo do algoritmo de Jacobi para autovalores, com uma tolerância $|a_{i,j}| \le 10^{-3}$.
		
		\begin{align*}
		%% ---------- %%
		\vec{A}^{(0)} &= \begin{bmatrix}
		3 & 2 & 0\\
		2 & 3 & -1\\
		0 & -1 & 3
		\end{bmatrix}~
		\vec{X}^{(0)} = \begin{bmatrix}
		1 & 0 & 0\\
		0 & 1 & 0\\
		0 & 0 & 1
		\end{bmatrix}\\[1cm]
		%% ---------- %%
		\vec{A}^{(1)} &= \begin{bmatrix}
		1 & 0 & 0.707\\
		0 & 5 & -0.707\\
		0.707 & -0.707 & 3
		\end{bmatrix}~
		\vec{X}^{(1)} = \begin{bmatrix}
		0.707 & 0.707 & 0\\
		-0.707 & 0.707 & 0\\
		0 & 0 & 1
		\end{bmatrix}\\[1cm]
		%% ---------- %%
		\vec{A}^{(2)} &= \begin{bmatrix}
		1 & 0.674 & 0.214\\
		0.674 & 2.775 & 0\\
		0.214 & 0 & 5.225
		\end{bmatrix}~
		\vec{X}^{(2)} = \begin{bmatrix}
		0.707 & 0.214 & -0.674\\
		-0.707 & 0.214 & -0.674\\
		0 & 0.953 & 0.303
		\end{bmatrix}\\[1cm]
		%% ---------- %%
		\vec{A}^{(3)} &= \begin{bmatrix}
		0.773 & 0 & 0.203\\
		0 & 3.002 & 0.068\\
		0.203 & 0.068 & 5.225
		\end{bmatrix}~
		\vec{X}^{(3)} = \begin{bmatrix}
		0.602 & 0.429 & -0.674\\
		-0.738 & -0.023 & -0.674\\
		-0.304 & 0.903 & 0.303
		\end{bmatrix}\\[1cm]
		%% ---------- %%
		\vec{A}^{(4)} &= \begin{bmatrix}
		0.764 & -0.003 & 0\\
		-0.003 & 3.002 & 0.068\\
		0 & 0.068 & 5.234
		\end{bmatrix}~
		\vec{X}^{(4)} = \begin{bmatrix}
		0.632 & 0.429 & -0.646\\
		-0.707 & -0.023 & -0.707\\
		-0.317 & 0.903 & 0.289
		\end{bmatrix}\\[1cm]
		%% ---------- %%
		\vec{A}^{(5)} &= \begin{bmatrix}
		0.764 & 0 & 0\\
		0 & 3.002 & 0.068\\
		0 & 0.068 & 5.234
		\end{bmatrix}~
		\vec{X}^{(5)} = \begin{bmatrix}
		0.632 & 0.428 & -0.646\\
		-0.707 & -0.022 & -0.707\\
		-0.316 & 0.904 & 0.289
		\end{bmatrix}\\[1cm]
		%% ---------- %%
		\vec{A}^{(6)} &= \begin{bmatrix}
		0.764 & 0 & 0\\
		0 & 3 & 0\\
		0 & 0 & 5.236
		\end{bmatrix}~
		\vec{X}^{(6)} = \begin{bmatrix}
		0.632 & 0.447 & -0.632\\
		-0.707 & 0 & -0.707\\
		-0.316 & 0.894 & 0.316
		\end{bmatrix}
		%% ---------- %%
		\end{align*}
		
		Após 6 iterações, temos os autovalores aproximados de $\vec{A}$ nas entradas da diagonal principal de $\vec{A}^{(6)}$, e seus respectivos autovetores nas respectivas colunas de $\vec{X}^{(6)}$.
		
		\subsubquest Vamos resolver agora o sistema $\vec{A}\vec{x} = \vec{b}$ de quatro maneiras distintas.
		
		\begin{enumerate}[wide, leftmargin=80pt]
			\item[1.: \textit{Cholesky}]~\\
			
			O método de \textit{Cholesky} nos dá uma fórmula direta para a fatoração $\vec{A} = \vec{L}\vec{L}^\T$:
			$$\vec{L} = {\begin{bmatrix}
				{\sqrt {\vec{A}_{1,1}}} & 0 & 0\\
				\frac{\vec{A}_{2,1}}{\vec{L}_{1,1}} & {\sqrt {\vec{A}_{2,2}-\vec{L}_{2,1}^{2}}} & 0\\
				\frac{\vec{A}_{3,1}}{\vec{L}_{1,1}} & \frac{\left(\vec{A}_{3,2}-\vec{L}_{3,1}\vec{L}_{2,1}\right)}{L_{2,2}} &{\sqrt {\vec{A}_{3,3}-\vec{L}_{3,1}^{2}-\vec{L}_{3,2}^{2}}}
			\end{bmatrix}}$$
			Portanto,
			$$\vec{L} = {\begin{bmatrix}
				\sqrt{3} & 0& 0\\
				\frac{2}{\sqrt{3}} & \sqrt {\frac{5}{3}} & 0\\
				0 &- \sqrt {\frac{3}{5}} & 2\sqrt {\frac{3}{5}}
				\end{bmatrix}}$$
			Resolvemos primeiro o sistema $\vec{L}\vec{y} = \vec{b}$:
			$$\begin{bmatrix}
			\sqrt{3} & 0& 0\\
			\frac{2}{\sqrt{3}} & \sqrt {\frac{5}{3}} & 0\\
			0 &- \sqrt {\frac{3}{5}} & 2\sqrt {\frac{3}{5}}
			\end{bmatrix} \begin{bmatrix}
			\vec{y}_1\\
			\vec{y}_2\\
			\vec{y}_3
			\end{bmatrix} = \begin{bmatrix}
			1\\
			-1\\
			1
			\end{bmatrix}
			$$
			de onde tiramos que $$\vec{y} = \begin{bmatrix}
			\frac{1}{\sqrt{3}}\\
			-\sqrt{\frac{5}{3}}\\
			0
			\end{bmatrix}$$
			
			Por fim, resolvemos $\vec{L}^\T \vec{x} = \vec{y}$:
			$$\begin{bmatrix}
			\sqrt{3} & \frac{2}{\sqrt{3}}& 0\\
			0 & \sqrt {\frac{5}{3}} & - \sqrt {\frac{3}{5}}\\
			0 & 0 & 2\sqrt {\frac{3}{5}}
			\end{bmatrix} \begin{bmatrix}
			\vec{x}_1\\
			\vec{x}_2\\
			\vec{x}_3
			\end{bmatrix} = \begin{bmatrix}
			\frac{1}{\sqrt{3}}\\
			-\sqrt{\frac{5}{3}}\\
			0
			\end{bmatrix}
			$$
			e assim temos $$\vec{x} = \begin{bmatrix}
			1\\
			-1\\
			0
			\end{bmatrix}$$
			
			\item[2.: \textit{Jacobi}]~\\
			
			\item[3.: \textit{Gauss-Seidel}]~\\
			
			\item[4.: Autovalores e autovetores]~\\
			
			Como $\vec{A}$ é simétrica, vale que $\vec{x} = \vec{\Theta}\vec{\lambda}^{-1}\vec{\Theta}^\T \vec{b}$, onde $\vec{\lambda}$ é a matriz diagonal dos autovalores e $\vec{\Theta}$ é a matriz dos autovetores. Logo,
			
			\begin{align*}
				\vec{x} = \left[\begin{array}{@{}ccc@{}}
				1 & 1 & 1 \\
				\frac{-\sqrt{5}}{2} & 0 & \frac{\sqrt{5}}{2}\\
				\frac{-1}{2} & 2 & \frac{-1}{2}
				\end{array}\right]
				\left[\begin{array}{@{}ccc@{}}
				\frac{1}{3 - \sqrt{5}} & 0 & 0 \\
				0 & \frac{1}{3} & 0\\
				0 & 0 & \frac{1}{3 + \sqrt{5}}
				\end{array}\right]
				\left[\begin{array}{@{}ccc@{}}
				1 & \frac{-\sqrt{5}}{2} & \frac{-1}{2} \\
				1 & 0 & 2\\
				1 & \frac{\sqrt{5}}{2} & \frac{-1}{2}
				\end{array}\right]
				\left[\begin{array}{@{}c@{}}
				1 \\
				-1 \\
				1 
				\end{array}\right] = \left[\begin{array}{@{}c@{}}
				1 \\
				-1 \\
				0 
				\end{array}\right] 
			\end{align*}
		\end{enumerate}
		
		\subsubquest Sabemos que, para uma matriz $\vec{A} \in \mathbb{C}^n$ temos $$\det\left(\vec{A}\right) = \prod_{i=1}^{n} \lambda_i$$
		Portanto, $\det\left(\vec{A}\right) = 3 \cdot (3 - \sqrt{5}) \cdot (3 + \sqrt{5}) = 12$.
	
	\quest%%4
	
	\begin{fortran}[Saída do Programa]
	 A:
	|   3.00000    2.00000    0.00000|
	|   2.00000    3.00000   -1.00000|
	|   0.00000   -1.00000    3.00000|
	b:
	|   1.00000|
	|  -1.00000|
	|   1.00000|
	DET =   12.000000000000000     
	RAIO ESPECTRAL =   5.2360679804753349     
	:: Decomposição LU (sem pivoteamento) ::
	L:
	|   1.00000    0.00000    0.00000|
	|   0.66667    1.00000    0.00000|
	|   0.00000   -0.60000    1.00000|
	U:
	|   3.00000    2.00000    0.00000|
	|   0.00000    1.66667   -1.00000|
	|   0.00000    0.00000    2.40000|
	y:
	|   1.00000|
	|  -1.66667|
	|   0.00000|
	x:
	|   1.00000|
	|  -1.00000|
	|   0.00000|
	:: Decomposição PLU (com pivoteamento) ::
	P:
	|   1.00000    0.00000    0.00000|
	|   0.00000    1.00000    0.00000|
	|   0.00000    0.00000    1.00000|
	L:
	|   1.00000    0.00000    0.00000|
	|   0.66667    1.00000    0.00000|
	|   0.00000   -0.60000    1.00000|
	U:
	|   3.00000    2.00000    0.00000|
	|   0.00000    1.66667   -1.00000|
	|   0.00000    0.00000    2.40000|
	y:
	|   1.00000|
	|  -1.66667|
	|   0.00000|
	x:
	|   1.00000|
	|  -1.00000|
	|   0.00000|
	DET
	12.000000000000000     
	x:
	|   1.00000|
	|  -1.00000|
	|   0.00000|
	:: Decomposição de Cholesky ::
	L:
	|   1.73205    0.00000    0.00000|
	|   1.15470    1.29099    0.00000|
	|   0.00000   -0.77460    1.54919|
	y:
	|   0.57735|
	|  -1.29099|
	|  -0.00000|
	x:
	|   1.00000|
	|  -1.00000|
	|  -0.00000|
	:: Método de Jacobi ::
	Matriz mal-condicionada.
	:: Método de Gauss-Seidel ::
	A:
	|   3.00000    2.00000    0.00000|
	|   2.00000    3.00000   -1.00000|
	|   0.00000   -1.00000    3.00000|
	x:
	|   1.00000|
	|  -1.00000|
	|  -0.00000|
	b:
	|   1.00000|
	|  -1.00000|
	|   1.00000|
	e =    6.2803698347351007E-016
	:: Método das Potências (Power Method) ::
	x:
	|   1.00000|
	|   1.11803|
	|  -0.50000|
	lambda:
	5.2360680332272143     
	:: Método de autovalores de Jacobi ::
	L:
	|   0.76393    0.00000    0.00000|
	|   0.00000    3.00000   -0.00000|
	|   0.00000   -0.00000    5.23607|
	X:
	|   0.63246    0.44721   -0.63246|
	|  -0.70711    0.00000   -0.70711|
	|  -0.31623    0.89443    0.31623|
	\end{fortran}

	\pagebreak
	\appendixpage
	\appendix \section*{Código}
	\lstinputlisting[style=fortranstyle, gobble=0]{matrixlib.f95}
	
%	\begin{thebibliography}{10}
%		
%	\end{thebibliography}
\end{document}
