\documentclass{homework}

\usepackage{homework}

\title{CPS740 - Prova 1}
\author{Pedro Maciel Xavier}
\register{116023847}


\begin{document}
	
	\maketitle
	
	\quest
	
	\begin{enumerate}[label=\textbf{\arabic*)}]
		\item Se $G$ é 3-colorível, então $G$ possui um ciclo ímpar. \textbf{Verdadeiro.} \par
		
		O \textbf{Teorema 5.1}\cite{jayme:18} afirma que $\rchi(G) = \min\{\rchi(\alpha_{v, w}(G)), \rchi(\beta_{v, w}(G))\}$ quando, tendo $(v, w) \notin E(G)$, $\alpha_{v, w}(G)$ é o grafo obtido pela inclusão da aresta $(v, w)$ ao grafo $G$ e $\beta_{v, w}(G)$ é aquele obtido pela identificação do vértice $v$ com o vértice $w$. Segue do teorema que $\rchi(G)$ é o tamanho do menor grafo completo encontrado através das aplicações recursivas da relação acima. Vale lembrar que como condição de parada, temos que $ \rchi(G) = r$ se $G = K_r$. \par
		
		Como $\rchi(G) = 3$, podemos afirmar que $G$ possui ao menos uma clique $K_3$ como subgrafo e, portanto, um ciclo ímpar de tamanho 3. \par
		
		\item Seja $G(V_1 \cup V_2, E)$ um grafo bipartido conexo. Então o grafo complementar $G^{c}$ também é conexo. \textbf{Falso.} \par
		
		Como $V_1$ e $V_2$ são conjuntos independentes, é evidente que o grafo complementar contará com duas cliques, formadas pelos vértices de $V_1$ e $V_2$, respectivamente. \par
		
		\shorthandoff{"}
		\begin{figure}[H]
			\centering
			\input{cps740-p1-t1.tikz}
			\caption{O Grafo $G(V_1 \cup V_2, E) = K_{|V_1|, |V_2|}$ e seu complemento $G^{c}$}
			\label{fig:1.2.1}
		\end{figure}
		\shorthandon{"}
		
		No entanto, no caso do grafo bipartido completo $K_{|V_1|, |V_2|}$ temos que seu complemento não possui nenhuma aresta que ligue um vértice de $V_1$ a algum outro em $V_2$. Temos assim um contraexemplo. \par
		
	\item Se retirarmos uma aresta qualquer do grafo $K_{3,3}$, o grafo resultante é planar. \textbf{Verdadeiro.} \par
	
	O \textbf{Teorema 2.7}\cite{jayme:18} afirma que um grafo é planar se e somente se não possuir nenhum subgrafo que seja uma subdivisão de $K_5$ ou $K_{3,3}$. A subdivisão de um grafo $G$ consiste em inserir um vértice $u \notin V(G)$ ao grafo após a remoção de uma aresta $(v, w) \in E(G)$ seguida da adição das arestas $(v, u)$ e $(u, w)$. \par
	
	Seja $\pi_{v, w}(G)$ uma subdivisão do grafo $G$ através da aresta $(v, w)$. Sabemos, pela construção deste processo, que $|V(\pi_{v, w}(G))| = |V(G)| + 1$ e que $|E(\pi_{v, w}(G))| = |E(G)| + 1$. Conclui-se que, sendo $\pi(G)$ uma subdivisão qualquer de $G$, temos que $|E(\pi(G))| \ge |E(G)|$. Seja $G$ o grafo obtido removendo uma aresta de $K_{3, 3}$, temos que todo subgrafo $H$ de $G$ possui $|E(H)| \le |E(G)|$. Como $|H(G)| \le |E(G)| < |E(K_{3, 3})|$, é claro que nenhum subgrafo $H$ de $G$ é subdivisão de $K_{3, 3}$. \par
	
	Da mesma forma, nenhum subgrafo próprio de $K_{3, 3}$ é capaz de conter uma subdivisão de $K_5$, já que pra isso é necessário possuir ao menos 5 vértices de grau 5. Portanto, qualquer grafo $G = K_{3, 3} - (v, w)$ tal que $(v, w) \in E(K_{3, 3})$ é planar. \par
	
	\shorthandoff{"}
	\begin{figure}[H]
		\centering
		\input{cps740-p1-t3.tikz}
		\caption{Um grafo bipartido planar}
		\label{fig:1.3.1}
	\end{figure}
	\shorthandon{"}
	
	Outro raciocínio possível seria verificar pela remoção de uma das arestas que se obtêm um grafo planar. Em seguida, basta observar que todos os grafos obtidos pela remoção de uma aresta de $K_{3, 3}$ são isomorfos e, portanto, planares. \par
	
	\item Seja $f$ um isomorfismo de um grafo $G$ para um grafo $H$, e seja $w$ um vértice em $G$. O grau de $w$ em $G$ é igual ao grau de $f(w)$ em $H$. \textbf{Verdadeiro.} \par
	
	Dados dois grafos, $G$ e $H$, dizemos que $G \cong H$ se $\exists f : V(G) \to V(H)$ tal que
		\begin{align}
			(w, v) \in E(G) \iff (f(w), f(v)) \in E(H)
		\end{align}
	Satisfeito, $f$ é dito um isomorfismo entre $G$ e $H$. Sabemos, portanto, que para um vértice qualquer $w \in V(G)$ tendo $f(w) \in V(H)$,
		\begin{align}
		\text{Seja } \mathbb{I}_{\Omega}\{\omega\} &\triangleq \begin{cases}
		1 \text{ se } \omega \in \Omega\\
		0 \text{ caso contrário}
		\end{cases} \nonumber \\
		~ \nonumber \\
		\text{grau}(w) &= \sum_{v \in V(G)} \mathbb{I}_{E(G)}\{(w, v)\}\\
					   &= \sum_{f(v) \in V(H)} \mathbb{I}_{E(H)}\{(f(w), f(v))\}\\
					   &= \text{grau}(f(w)) & \cqd \nonumber
		\end{align}
	De $(2)$ para $(3)$ utilizamos a relação $(1)$, extraída da definição de isomorfismo em grafos presente no Capítulo 2 do livro\cite{jayme:18}.
	
	\item O Grafo abaixo é planar: \textbf{Falso.} \par
	
	Seja $\pi(G)$ uma subdivisão qualquer de $G$. Invocando mais uma vez o \textbf{Teorema 2.7}\cite{jayme:18}, vamos buscar por subdivisões de $K_5$ e $K_{3, 3}$ no grafo. Certamente não há nenhuma instância de $\pi(K_5)$, visto que só existe um vértice que possui grau maior ou igual a $5$, quando são necessários ao menos $5$.
	
	\shorthandoff{"}
	\begin{figure}[H]
		\centering
		\input{cps740-p1-t2.tikz}
		\caption{Grafo que não é planar pois esconde em si o $K_{3,3}$.}
		\label{fig:1.5.1}
	\end{figure}
	\shorthandon{"}
	
	No entanto, encontramos uma subdivisão de $K_{3, 3}$ e podemos afirmar que o grafo não é planar.
	
	\item Todo hipercubo de dimensão $n$, $n \ge 1$, possui ciclo Hamiltoniano. \textbf{Falso.}
	
	Seja $\mathscr{H}_n$ o $n$-ésimo hipercubo, um grafo com $2^n$ vértices de grau $n$. \par
	
	Construímos $\mathscr{H}_k$ utilizando duas cópias de $\mathscr{H}_{k-1}$, notadamente $\mathscr{H}_{k-1}$ e $\mathscr{H}_{k-1}'$, ligando os vértices das duas componentes através da adição de arestas $(w, f(w)), w \in \mathscr{H}_{k-1}, f(w) \in \mathscr{H}_{k-1}'$ definidas pelo isomorfismo trivial $f:\mathscr{H}_{k-1}\to\mathscr{H}_{k-1}'$. O processo é ilustrado pela figura \ref{fig:1.6.1}. \par

	\shorthandoff{"}
	\begin{figure}[H]
		\centering
		\input{cps740-p1-t4.tikz}
		\caption{A construção de um Hipercubo}
		\label{fig:1.6.1}
	\end{figure}
	\shorthandon{"}

	Assim, o método descrito acima, que constrói e define a família de grafos $\mathscr{H} = \{\mathscr{H}_{k} : k \in \mathbb{N}\}$, pode ser descrito de maneira aritmética e geométrica para além da intuição topológica presente na figura.\par
	
	Comecemos com o conjunto de vértices $\mathscr{V}_1 = \{0, 1\}$ representados por vetores binários de tamanho $1$. Para obter o conjunto $\mathscr{V}_k = V(\mathscr{H}_k)$, basta calcular $\mathscr{V}_k = \mathscr{V}_{k-1} \times \mathscr{V}_1$. O conjunto $E(\mathscr{H}_n)$ de arestas do hipercubo é definido por $\mathscr{E}_n = \{(v, w) : v, w \in \mathscr{V}_n, v \oplus w = 2^{m}, m \in \mathbb{N} \cup \{0\}\}$, onde $x \oplus y$ denota a operação \texttt{xor} (ou-exclusivo) \textit{bit}-a-\textit{bit}. Esta representação é útil pois indica uma imersão natural dos vértices de $\mathscr{V}_n$ em $[0, 1]^{n} \subseteq\mathbb{R}^n$.\par

	Vamos utilizar como caso base de indução o hipercubo $\mathscr{H}_2$, que é um ciclo hamiltoniano por si só. Como passo de indução, supomos que um hipercubo $\mathscr{H}_{k-1}$ possui um ciclo hamiltoniano $\mathscr{C}_{k-1}$ do qual uma aresta $(v, w)$ faz parte. Fazemos uma cópia $\mathscr{H}_{k-1}'$ de $\mathscr{H}_{k-1}$ para construir $\mathscr{H}_{k}$ segundo o processo já explicitado. Para compor o ciclo $\mathscr{C}_k$, vamos remover a aresta $(v, w)$ de $\mathscr{C}_{k-1}$ assim como a aresta $(v', w')$ de $\mathscr{C}_{k-1}'$ (ciclo Hamiltoniano em $\mathscr{H}_{k-1}'$). Por fim, adicionamos as arestas $(v, v')$ e $(w, w')$ ao subgrafo $\mathscr{C}_{k-1} \cup \mathscr{C}_{k-1}'$ para obter $\mathscr{C}_{k}$. \par

	\shorthandoff{"}
	\begin{figure}[H]
		\centering
		\input{cps740-p1-t5.tikz}
		\caption{Ciclo Hamiltonianos em Hipercubos}
		\label{fig:1.6.2}
	\end{figure}
	\shorthandon{"}
	
	Com este argumento, fica demonstrado que todo hipercubo $\mathscr{H}_{n}, n \ge 2$ possui ciclo Hamiltoniano. No entanto, isso não vale para $n = 1$, já que $\mathscr{H}_{1}$ possui apenas $2$ vértices e não é capaz de conter ciclos. \par

	\end{enumerate}

	\quest Vamos supor que utilizamos $r_1$ cores em uma coloração ótima para $G_1$ e que precisamos de ao menos $r_2$ cores para colorir $G_2$. Seja $r = \max \{r_1, r_2\} = \max\{\rchi(G_1), \rchi(G_2)\}$, vamos usar no máximo $r$ cores para pintar os dois grafos separadamente. \par
	
	Sejam $C_1 : V(G_1) \to [r]\footnote{Notação para o conjunto dos números naturais até $n$, $[n] = \{1, 2, \dots, n\}$}, C_2 : V(G_2) \to [r]$ colorações para os vértices de $G_1$ e $G_2$, respectivamente. Chamaremos $G = \sigma_{v, w}(G_1, G_2)$ o grafo resultante da identificação dos vértices $v \in V(G_1)$ e $w \in V(G_2)$, escolhidos ao acaso, em um único vértice $s \in V(G)$. Diante dessas informações temos os seguintes casos possíveis: \par
	
	\shorthandoff{"}
	\begin{figure}[H]
		\centering
		\input{cps740-p1-t6.tikz}
		\caption{Coloração ótima de cada grafo.}
		\label{fig:2.1}
	\end{figure}
	\shorthandon{"}
	
	\begin{enumerate}[label=\Roman*.]
		\item Se $C_1(v) = C_2(w) = r^{\ast} \in [r]$, podemos identificar os vértices sem alterar a coloração de cada grafo. Isto é, construir a coloração $C : V(G) \to [r]$ através de $C_1$ e $C_2$: \par
		
		$$C(u) \triangleq \begin{cases}
		C_1(u) \text{ se } u \in V(G_1)\\
		C_2(u) \text{ se } u \in V(G_2)\\
		r^{\ast} \text{ caso contrário}
		\end{cases}$$
		
		\shorthandoff{"}
		\begin{figure}[H]
			\centering
			\input{cps740-p1-t7.tikz}
			\caption{Identificação de vértices de mesma cor.}
			\label{fig:2.2}
		\end{figure}
		\shorthandon{"}
		
		Dessa maneira, é garantido que não será necessária nenhuma cor que não esteja em $[r]$, ou seja, $\rchi(\sigma_{v, w}(G_1, G_2,)) \le r$.
		
		\textbf{Teorema}\\
		Seja $H$ subgrafo de $G$, então $\rchi(H) \le \rchi(G)$
		
		\textbf{Prova}\\
		Suponha que $G$ possui uma coloração ótima $ C : V(G) \to [\rchi(G)] $, basta aplicar a mesma coloração para $H$, e esta será também uma coloração válida para $H$. $\cqd$
		
		\item Caso $C_1(v) \neq C_1(w)$, vamos construir uma coloração alternativa $C_2'$ para $G_2$. Queremos que $C_2(w)' = C_1(v)$. Vamos, portanto, colorir os vértices cuja cor é a mesma de $v$ com a cor de $w$, a medida que colorimos aqueles que tem a cor de $w$ com a cor de $v$. \par
		
		$$C_2'(u) \triangleq \begin{cases}
		C_2(v) \text{ se } C_2(u) = C_2(w)\\
		C_2(w) \text{ se } C_2(u) = C_2(v)\\
		C_2(u) \text{ caso contrário}
		\end{cases}$$
		
		\shorthandoff{"}
		\begin{figure}[H]
			\centering
			\input{cps740-p1-t8.tikz}
			\caption{Identificação de vértices de cores distintas.}
			\label{fig:2.3}
		\end{figure}
		\shorthandon{"}
		
		A coloração $C$ é construída pelo mesmo método do caso {I.}, mas desta vez a partir de $C_1$ e $C_2'$. Recolorir o grafo segundo este método eventualmente introduz uma nova cor ao grafo recolorido, ao custo da exclusão de uma de suas cores, fazendo com que o número de cores utilizadas seja preservado.\par
		
		Portanto, este segundo caso é reduzido ao primeiro, para o qual já temos um argumento para afirmar que $\rchi(\sigma_{v, w}(G_1, G_2)) = \max \{\rchi(G_1), \rchi(G_2)\}$. \par
		
	\end{enumerate}
	
	\quest Seja $S \subseteq \mathbb{R}^2$ um conjunto de $n$ pontos sobre o plano.
	
	
	
	
	
	
	\begin{thebibliography}{10}
		\bibitem{jayme:18} SZWARCFITER, Jayme Luiz, \textbf{Teoria Computacional de Grafos}, 1ª edição, Rio de Janeiro, 2018.
	\end{thebibliography}
	
\end{document}
