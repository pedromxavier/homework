\documentclass{homework}
\usepackage{homework}

\title{CPS740 - Prova 2}
\author{Pedro Maciel Xavier}
\register{116023847}

\begin{document}
	
	\cmaketitle{cps740-p2-wallpaper.png}
	
	\quest
	
	\begin{fig}
		\input{cps740-p2-1.tikz}
	\end{fig}
	
	\subquest[Algoritmo de \textit{Djkstra}] %%1
	
	%%\begin{fig}
	\input{cps740-p2-1a.tikz}
	%%\end{fig}
	
	\subquest%%2
	
	\subquest%%3
	
	\subquest Independentemente do critério de ordenação na busca, uma árvore geradora oriunda de uma busca em largura optaria por atingir o vértice $C$ utilizando-se da aresta $(D, C)$, de custo $9$. A árvore de caminho mínimo, no entanto, chegaria ao vértice $C$ através de $B$, uma vez que o caminho $(D, B, C)$ possui distância total $2 + 1 = 3$.
	
	\subquest
	
	\subquest

	\quest

	\subquest
	
	\subquest
	
	\subquest
	
	\subquest	
	\begin{thebibliography}{10}
		\bibitem{jayme:18} SZWARCFITER, Jayme Luiz, \textbf{Teoria Computacional de Grafos}, 1ª edição, Rio de Janeiro, 2018.
	\end{thebibliography}
	\newpage
	\let\clearpage\relax
\end{document}
