\documentclass{homework}

\usepackage{homework}
\usepackage{pythontex}

\title{CPS740 - Lista 1}
\author{Pedro Maciel Xavier}
\register{116023847}

\begin{document}
	\maketitle
	
	\quest
	
	\subquest[Primeiro Algoritmo]
	
	\subsubquest[Algoritmo]
	
	\begin{algor}
seja L[n]

def ordena(L[], n):
	seja m, k
	para i = n até 2:
		m <- $-\infty$
		para j = i até 1:
			se L[j] > m:
				m <- L[j]
				k <- j
		L[j] <-> L[k]
	retorna L
	\end{algor}

	\subsubquest[Passo-a-passo]
	\begin{align*}
	L = &\left\{2,7,5,6,\mathbf{9},0,1,4,8,5,\mathbf{3}\right\} 1 \times 11 = 11\\
	&\left\{2,7,5,6,3,0,1,4,\mathbf{8},\mathbf{5},9\right\} 2 \times 10 = 20\\
	&\left\{2,\mathbf{7},5,6,3,0,1,4,\mathbf{5},8,9\right\} 3 \times  9 = 27\\
	&\left\{2,5,5,\mathbf{6},3,0,1,\mathbf{4},7,8,9\right\} 4 \times  8 = 32\\
	&\left\{2,5,\mathbf{5},4,3,0,\mathbf{1},6,7,8,9\right\} 5 \times  7 = 35\\
	&\left\{2,\mathbf{5},1,4,3,\mathbf{0},5,6,7,8,9\right\} 6 \times  6 = 36\\
	&\left\{2,0,1,\mathbf{4},\mathbf{3},5,5,6,7,8,9\right\} 7 \times  5 = 35\\
	&\left\{2,0,1,\mathbf{3},4,5,5,6,7,8,9\right\} 8 \times  4 = 32\\
	&\left\{\mathbf{2},0,\mathbf{1},3,4,5,5,6,7,8,9\right\} 9 \times  3 = 27\\
	&\left\{\mathbf{1},\mathbf{0},2,3,4,5,5,6,7,8,9\right\}10 \times  2 = 20\\
	&\left\{0,1,2,3,4,5,5,6,7,8,9\right\} \text{Total de $275$ passos}
	\end{align*}
	
	\subquest[Segundo Algoritmo]
	
	\subsubquest[Algoritmo]
	\begin{algor}
seja L[n]

def ordena(L[], n):
	
	\end{algor}

	%% 2
	\quest
	Seja $G = (V, E)$ um grafo com $m$ arestas e $n$ vértices.\\
	
	\subsubquest A \textbf{Matriz de Adjacências} $A^{n \times n}$ de $G$ é dada por:
		\[A_{i,j} = \begin{cases}
			1, \text{ se } (i, j) \in E\\
			0, \text{ caso contrário}
		\end{cases}\]
	Desta maneira a matriz terá suas $n^2$ entradas. Por isso, precisaremos de, no mínimo, $n^2$ \emph{bits} para representá-lo assim.\\
	
	\subsubquest A \textbf{Estrutura de Adjacências} $L$ de $G$ consiste em uma lista de tamanho $n$ contendo em cada nó $i$ um ponteiro para uma lista que contém os vértices adjacentes a $i$. Ou simplesmente:
		$$j \in L_i \iff (i, j) \in E$$
	Neste caso, a demanda por espaço não está diretamente relacionada ao número de vértices, mas sim, à quantidade de arestas. Grafos com poucas conexões podem ser representados de maneira eficiente com essa estrutura. Por outro lado, redes muito conexas tornam esta abordagem indesejável.\\
	\\
	Para cada aresta introduzida no grafo temos um custo de $2e$ \emph{bits}, onde $e$ representa o armazenamento de um nó $j$ em uma lista e depende da implementação. O custo dobrado vem da necessidade de armazenar cada aresta em duas listas distintas. Por fim, o custo total para armazenar o grafo $G$ desta maneira seria de $2e \times m + v \times n$, considerando $v$ o custo para armazenar $L$.\\
	
	\subsubquest Comparando as duas escolhas de representação, podemos optar pela \textbf{Matriz de Adjacências} sempre que
		$$n^2 < 2e \times m + v \times n$$
	%% 3
	\quest
	%% 4
	\quest
	%% 5
	\quest
	%% 6
	\quest
	%% 7
	\quest
	\subsubquest A soma dos quadrados dos números de $1$ até $n$.
	
	\subsubquest São executadas sucessivas multiplicações seguidas de soma.
	\subsubquest Temos $n$ multiplicações e $n$ somas, totalizando $2n$ operações, ou seja, tempo linear $O(n)$.
	\subsubquest Existem algoritmos melhores. Demonstração:\\
	Temos que
	\begin{align*}
		x^3 - (x-1)^3 &= x^3 - (x^3 - 3x^2 + 3x - 1)\\
					  &= x^3 - x^3 + 3x^2 - 3x + 1\\
					  &= 3x^2 - 3x + 1
	\end{align*}
	Assim,
	\begin{align*}
		&\implies& (x-1)^3 - (x-2)^3 &= 3(x-1)^2 - 3(x-1) + 1\\
		&\implies& [x^3 - (x-1)^3] + [(x-1)^3 - (x-2)^3] &= [3x^2 - 3x + 1] + [3(x-1)^2 - 3(x-1) + 1]\\
		&\implies& x^3 - (x-2)^3 &= 3[x^2 + (x-1)^2] - 3[x + (x-1)] + 2\\
		& & &~\vdots\\
		&\implies& x^3 - (x-(n+1))^3 &= 3\sum_{i=1}^{n}(x-i)^2 - 3\sum_{i=1}^n(x-i) + (n+1)
	\end{align*}
	Escolhendo $x = 0$:
	\begin{align*}
	&\implies& -(-(n+1))^3 &= 3\sum_{i=1}^{n}(-i)^2 - 3\sum_{i=1}^n(-i) + (n+1)\\
	&\implies&     (n+1)^3 &= 3\sum_{i=1}^{n} i^2 + 3\sum_{i=1}^n i + (n+1)\\
	&\implies& \sum_{i=1}^{n} i^2 &= \frac{(n+1)^3 - (n+1)}{3} - \sum_{i=1}^n i\\
	&        &					  &= \frac{(n+1)^3 - (n+1)}{3} - \frac{n (n+1)}{2}
	\end{align*}
	Desta maneira, temos 3 somas, 3 multiplicações e 2 divisões, totalizando 8 operações, ou seja, tempo constante $O(1)$.
\end{document}
